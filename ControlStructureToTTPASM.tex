\documentclass[9pt]{article}
\usepackage[utf8]{inputenc}
\usepackage{geometry}
\geometry{a4paper, margin=0.4in, top=0.3in, bottom=0.3in}
\usepackage{listings}
\usepackage{xcolor}
\usepackage{hyperref}
\usepackage{booktabs}
\usepackage{longtable}
\usepackage{enumitem}
\usepackage{titlesec}
\usepackage{parskip}
\usepackage{fancyhdr}

% Remove header/footer spacing
\pagestyle{plain}
\setlength{\headheight}{0pt}
\setlength{\headsep}{0pt}
\setlength{\footskip}{10pt}

% Compact spacing for sections
\titlespacing*{\section}{0pt}{3pt}{1pt}
\titlespacing*{\subsection}{0pt}{2pt}{0.5pt}
\titlespacing*{\subsubsection}{0pt}{1.5pt}{0pt}
\titlespacing*{\paragraph}{0pt}{0.5pt}{0pt}

% Compact lists
\setlist{nosep, leftmargin=12pt, itemsep=0pt, parsep=0pt, topsep=1pt, partopsep=0pt}
\setenumerate{nosep, leftmargin=12pt, itemsep=0pt, parsep=0pt, topsep=1pt}
\setitemize{nosep, leftmargin=12pt, itemsep=0pt, parsep=0pt, topsep=1pt}

% Reduce paragraph spacing
\setlength{\parskip}{1pt}
\setlength{\parindent}{0pt}
\setlength{\baselineskip}{10pt}

% Compact code listings - preserves ALL comments
\lstset{
    basicstyle=\ttfamily\scriptsize,
    breaklines=true,
    frame=single,
    framesep=2pt,
    numbers=left,
    numbersep=3pt,
    numberstyle=\tiny\color{gray},
    xleftmargin=5pt,
    xrightmargin=5pt,
    aboveskip=2pt,
    belowskip=2pt,
    showstringspaces=false,
    keepspaces=true,
    tabsize=2,
    commentstyle=\color{gray!70}\itshape,
    keywordstyle=\color{blue}\bfseries,
    stringstyle=\color{red!70},
    escapeinside={@*}{*@},
}

\title{Controlstructuretottpasm}
\author{}
\date{}

\begin{document}

\maketitle
\vspace{-10pt}

Module 0303: "Compiling" control structure to TTPASM

About this module

Purpose of this module

C control structures

Condition statements

Pre-checking loop

Post-checking loop

Conditional goto and boolean operators

not

or

and

Simple reductions

Comparisons

General compare

About nesting and labels

Indentation

More useful names

Sequential numbers

Nested constructs

Boolean operator reduction transformations

A longer example

\begin{enumerate}
\item About this module
\end{enumerate}
Prerequisites:

Objectives: This module explains how C control structures are compiled to assembly

language code.

\begin{enumerate}
\item Purpose of this module
\end{enumerate}
This module contains the instructions to convert a C program into assembly language code in a

step-by-step manner. While this module is not intended for people who want to write

compilers, it is certainly useful for that purpose.

\begin{enumerate}
\item C control structures
\end{enumerate}
This section introduces simple transformations that turn regular C code into C code that has

no control structures except for “conditional goto” constructs.

\begin{enumerate}
\item 1 Condition statements
\end{enumerate}
First, it is important to understand the C syntax of conditional statements without braces.

if (a<b)
a++;
b--;
The above code is actually the same as the following code.

if (a<b)
{
a++;
}
b--;
This is because, after the parenthesized condition, C++ expects a single statement to specify

the then-statement. a++;, by itself, is a statement. After the then-statement, there are two

alternatives.

the else keyword is encountered: then C++ expects another statement as the else-

statement.

the else keyword is not encountered: then C++ ends the conditional statement , treating

whatever that is following as a continuation after the conditional statement.

Block statements (braces {}) are highly recommended in normal C++ coding because of this.

However, in the context of this module, the intention is “flattening” the structured code. As a

result, the intention is to get rid of block statements one step at a time.

if (c)
{
blk1; // blk1 is a placeholder for the then-statement
}
else
{
blk2; // blk2 is a placeholder for the else-statement
}
Translates to

if (!c) goto L1;
blk1;
goto L2;
L1:
blk2;
L2:
If there is no else, then

if (c)
{
blk1;
}
translates to

if (!c) goto L1;
blk1;
L1:
This is because the following code optimizes to no code:

goto L2;
L2:
\begin{enumerate}
\item 2 Pre-checking loop
\end{enumerate}
while (c)
{
blk1; // blk1 is a placeholder for the code in repetition
}
Translates to

L1:

if (!c) goto L2;
blk1;
goto L1;
L2:
\begin{enumerate}
\item 3 Post-checking loop
\end{enumerate}
do
{
blk1; // blk1 is a placeholder for the code in repetition
}
while (c);
Translates to

L1:

blk1;
if (c) goto L1;
\begin{enumerate}
\item Conditional goto and boolean operators
\end{enumerate}
From the previous section, if a condition involves negation (not), disjunction (or) or conjunction

(and), we must first take care of those operators. This section discusses how those operators,

while used in conditional goto statements, can be transformed.

\begin{enumerate}
\item 1 not
\end{enumerate}
if (!c) goto L1;
Transforms to

if (c) goto L2;
goto L1;
L2:
\begin{enumerate}
\item 2 or
\end{enumerate}
if (c || d) goto L1;
becomes

if (c) goto L1;
if (d) goto L1;
\begin{enumerate}
\item 3 and
\end{enumerate}
if (c && d) goto L1;
becomes

if (!c) goto L2;
if (d) goto L1;
L2:
\begin{enumerate}
\item 4 Simple reductions
\end{enumerate}
Although the transformations will work in all cases, they are not always necessary.

For example, !(x >= y) is (x < y).

Also, (x < y) || (x > y) is (x != y).

Using algebra to reduce boolean operators saves a bit of control structure transformations,

and can lead to more efficient code.

\begin{enumerate}
\item Comparisons
\end{enumerate}
With the previous sections, regular C control structures are reduced to conditional goto

statements that do not use any boolean operators. This means that the conditional goto

statements must only contain comparison operators.

Because the toy processor can only confirm “less than” and “equal to” using a single

instruction, we need to translate all comparison operators to these two operators (with the help

of some logical operators):

is just that

is reversed to

is

is

is just that

is

\begin{enumerate}
\item 1 General compare
\end{enumerate}
if (x r y) goto L1;
is a generalized conditional goto statement. x r y is the abstraction of x is r y. In an

actual program, x is an expression, r is a comparison operator (the toy processor is

confined to equal-to and less-than), and y is another expression.

For now, however, let us assume that x and y are registers.

This conditional goto statement transforms to the following pseudo assembly instructions:

cmp x , y
jri L
x < y
x > y y < x
x ≤ y ( x < y )∨( x = y )
x ≥ y ( y < x )∨( y = x )
x = y
x ≠ y ¬( x = y )
For example, let us assume the C code is as follows (x and y are unsigned):

if (x < y) goto L1;
Then the matching assembly code is as follows:

cmp x,y
jci L
Because the toy processor can only compare registers, compare to constants need to first load

the constant to a register using the ldi instruction.

\begin{enumerate}
\item About nesting and labels
\end{enumerate}
Although the previous sections use the generic labels L0, L1 and L2, your program should

use more useful label names. This section recommends a structured method to name the

labels.

\begin{enumerate}
\item 1 Indentation
\end{enumerate}
While most assembly language programs do not make use of indentation, it makes sense to

use indentation the same way as in a high level programming language. This is because

indentation reflects the structure of an algorithm, and assembly language programs can be

well structured like a high level programming language.

\begin{enumerate}
\item 2 More useful names
\end{enumerate}
Use more useful names, such as loopBegin, loopEnd, else, endIf and etc. whenever it is

appropriate.

\begin{enumerate}
\item 3 Sequential numbers
\end{enumerate}
Constructs of the same nesting level should use sequential numbers as a suffix to differentiate

from each other. For example,

if (x >= y)
x++;
if (y >= x)
y++;
translates to

if (x < y) goto endIf1;
x++;
endIf1:
if (y < x) goto endIf2;
y++;
endIf2:
\begin{enumerate}
\item 4 Nested constructs
\end{enumerate}
Nested constructs should technically use a suffix or prefix notation to indicate how they are

nested. A prefix (where the deeper structure is named first) is easier to read, as the first part of

the label shows the meaning of the label. However, a suffix is more natural, especially to people

who are used to C/C++ programming.

Using the suffix method,

while (x < 3 )
{
if (y == x)
{
z++;
}
x++;
}

transforms to

beginLoop1:
if (x >= 3 ) goto endLoop1;
if (y != x) goto loop1_endIf1;
z++;
loop1_endIf1:
x++;
goto beginLoop1;
endLoop1:
The following represents the same logic, but using the prefix method (nested structure named

first):

beginLoop1:
if (x >= 3 ) goto endLoop1;
if (y != x) goto endIf1_loop1;
z++;
endIf1_loop1:
x++;
goto beginLoop1;
endLoop1:
Note that the “sequence number” of each nesting can restart from 1 because the prefix or

suffix of the parent structure makes it unique.

An alternative (to more experienced programmers) is to simply use sequential numbers for

each construct. As one construct is converted, simply pick the next available sequential

number. This method requires that the programmer only perform the translation of one

construct at a time. Furthermore, the label names no longer reflect the nested structure of the

code.

\begin{enumerate}
\item 5 Boolean operator reduction transformations
\end{enumerate}
Boolean operator reduction transformation are low level transformations. As such, the labels

generated as a result should simply add a suffix to the original labels. For example,

if ((x == y) && (x < 0 )) goto endIf2_loop5;
translates to

if (!(x == y)) goto endIf2\textit{loop5}1;
if (x < 0 ) goto endIf2_loop5;
endIf2\textit{loop5}1:
\begin{enumerate}
\item A longer example
\end{enumerate}
Here is a long(er) example to illustrate the concepts introduced in this module. To make the

“diff ” between revisions stand out more, you can copy-and-paste each section of code into

files. Then, use vimdiff or a similar visualized diff command to have the changes

highlighted. If you have XWindow running, there is a whole list of these tools: tkdiff, xxdiff,

diffuse, gvimdiff, etc.

Let’s say we want to convert the following code into assembly code:

int x,y,z;
int main (void)
{
x = 5 ;
z = 0 ;
while (x > 0 )
{
y = 0 ;
while (y < 5 )
{
if (((x > 2 ) && (x != y)) || (y == 1 ))
{
z = z + x;
}
}
}
}
This program no particular “meaning” as in useful behavior, its only purpose is to serve as an

example.

First, I can comment out the entire program, and start to work on it little-by-little. This can be

done with a single command in vi, there is a reason to learn how to use it!

// int x,y,z;
// int main(void)
// {
// x = 5;
// z = 0;
// while (x > 0)
// {
// y = 0;
// while (y < 5)
// {
// if (((x > 2) && (x != y)) || (y == 1))
// {
// z = z + x;
// }
// }
// }
// }
Next, let us declare the global variables:

// int x,y,z;
// int main(void)
// {
// x = 5;
// z = 0;
// while (x > 0)
// {
// y = 0;
// while (y < 5)
// {
// if (((x > 2) && (x != y)) || (y == 1))
// {
// z = z + x;
// }
// }
// }
// }
halt
x:
byte 0
y:
byte 0
z:
byte 0
The initialization portion is also quite easy.

// int x,y,z;
// int main(void)
// {
// x = 5;
ldi a,x
ldi b, 5
st (a),b
// z = 0;
ldi a,z
ldi b, 0 ;
st (a),b
// while (x > 0)
// {
// y = 0;
// while (y < 5)
// {
// if (((x > 2) && (x != y)) || (y == 1))
// {
// z = z + x;
// }

// }

// }

// }

halt
x:
byte 0
y:
byte 0
z:
byte 0
Adding x to z requires an intermediate register, but otherwise it is also straightforward.

// int x,y,z;
// int main(void)
// {
// x = 5;
ldi a,x
ldi b, 5
st (a),b
// z = 0;
ldi a,z
ldi b, 0 ;
st (a),b
// while (x > 0)
// {
// y = 0;
// while (y < 5)
// {
// if (((x > 2) && (x != y)) || (y == 1))
// {
// z = z + x;
ldi a,x // a = &x
ld a,(a) // a = x
ldi b,z // b = &z
ld b,(b) // b = z
add a,b // a = x+z
ldi b,z // b = &z
st (b),a // z = x+z
// }
// }
// }

// }

halt
x:
byte 0
y:
byte 0
z:
byte 0
Now let us take care of the inner-most conditional statement. First, we reduce to “ugly”

conditional statement code:

// int x,y,z;
// int main(void)
// {
// x = 5;
ldi a,x
ldi b, 5
st (a),b
// z = 0;
ldi a,z
ldi b, 0 ;
st (a),b
// while (x > 0)
// {
// y = 0;
// while (y < 5)
// {
// if (((x > 2) && (x != y)) || (y == 1))
// if (!(((x > 2) && (x != y)) || (y == 1))) goto while1\textit{while1}endif1;
// {
// z = z + x;
ldi a,x // a = &x
ld a,(a) // a = x
ldi b,z // b = &z
ld b,(b) // b = z
add a,b // a = x+z
ldi b,z // b = &z
st (b),a // z = x+z
// }
DeMorgan’s law allows us to simplify the negated disjunction a little. DeMorgan’s law says

.

while1\textit{while1}endif1:
// }
// }
// }
halt
x:
byte 0
y:
byte 0
z:
byte 0
// int x,y,z;
// int main(void)
// {
// x = 5;
ldi a,x
ldi b, 5
st (a),b
// z = 0;
ldi a,z
ldi b, 0 ;
st (a),b
// while (x > 0)
// {
// y = 0;
// while (y < 5)
// {
// if (((x > 2) && (x != y)) || (y == 1))
// if (!(((x > 2) && (x != y)) || (y == 1))) goto while1\textit{while1}endif1;
// if ((!((x > 2) && (x != y)) && !(y == 1))) goto while1\textit{while1}endif1;
// {
// z = z + x;
ldi a,x // a = &x
ld a,(a) // a = x
ldi b,z // b = &z
ld b,(b) // b = z
add a,b // a = x+z
X + ̄ Y = X ̄⋅ Y ̄
Now is a good time to take care of the outer conjunction. We can reduce the conjunction by

converting the conditional-goto into multiple statements:

ldi b,z // b = &z
st (b),a // z = x+z
// }
while1\textit{while1}endif1:
// }
// }
// }
halt
x:
byte 0
y:
byte 0
z:
byte 0
// int x,y,z;
// int main(void)
// {
// x = 5;
ldi a,x
ldi b, 5
st (a),b
// z = 0;
ldi a,z
ldi b, 0 ;
st (a),b
// while (x > 0)
// {
// y = 0;
// while (y < 5)
// {
// if (((x > 2) && (x != y)) || (y == 1))
// if (!(((x > 2) && (x != y)) || (y == 1))) goto while1\textit{while1}endif1;
// if ((!((x > 2) && (x != y)) && !(y == 1))) goto while1\textit{while1}endif1;
// if ((x > 2) && (x != y)) goto while1\textit{while1}if1;
// if (!(y == 1)) goto while1\textit{while1}endif1;
// {
// z = z + x;
We can now take care of the simple transformations:

while1\textit{while1}if1:
ldi a,x // a = &x
ld a,(a) // a = x
ldi b,z // b = &z
ld b,(b) // b = z
add a,b // a = x+z
ldi b,z // b = &z
st (b),a // z = x+z
// }
while1\textit{while1}endif1:
// }
// }
// }
halt
x:
byte 0
y:
byte 0
z:
byte 0
// int x,y,z;
// int main(void)
// {
// x = 5;
ldi a,x
ldi b, 5
st (a),b
// z = 0;
ldi a,z
ldi b, 0 ;
st (a),b
// while (x > 0)
// {
// y = 0;
// while (y < 5)
// {
// if (((x > 2) && (x != y)) || (y == 1))
// if (!(((x > 2) && (x != y)) || (y == 1))) goto while1\textit{while1}endif1;
// if ((!((x > 2) && (x != y)) && !(y == 1))) goto while1\textit{while1}endif1;
Now we translate the other conjunction.

// if ((x > 2) && (x != y)) goto while1\textit{while1}if1;
// if (!(y == 1)) goto while1\textit{while1}endif1;
ldi a,y // a = &y
ld a,(a) // a = y
ldi b, 1 // b = 1
cmp a,b // a-b
jzi while1\textit{while1}if
jmpi while1\textit{while1}endif
// {
// z = z + x;
while1\textit{while1}if1:
ldi a,x // a = &x
ld a,(a) // a = x
ldi b,z // b = &z
ld b,(b) // b = z
add a,b // a = x+z
ldi b,z // b = &z
st (b),a // z = x+z
// }
while1\textit{while1}endif1:
// }
// }
// }
halt
x:
byte 0
y:
byte 0
z:
byte 0
// int x,y,z;
// int main(void)
// {
// x = 5;
ldi a,x
ldi b, 5
st (a),b
// z = 0;
ldi a,z
ldi b, 0 ;
st (a),b
// while (x > 0)
// {
// y = 0;
// while (y < 5)
// {
// if (((x > 2) && (x != y)) || (y == 1))
// if (!(((x > 2) && (x != y)) || (y == 1))) goto while1\textit{while1}endif1;
// if ((!((x > 2) && (x != y)) && !(y == 1))) goto while1\textit{while1}endif1;
// if ((x > 2) && (x != y)) goto while1\textit{while1}if1;
// if (!(x > 2)) goto while1\textit{while1}if1_l1;
// if ((x < 2) || (x == 2)) goto while1\textit{while1}if1_l1;
// if (x != y) goto while1\textit{while1}if1;
while1\textit{while1}if1_l1:
// if (!(y == 1)) goto while1\textit{while1}endif1;
ldi a,y // a = &y
ld a,(a) // a = y
ldi b, 1 // b = 1
cmp a,b // a-b
jzi while1\textit{while1}if
jmpi while1\textit{while1}endif
// {
// z = z + x;
while1\textit{while1}if1:
ldi a,x // a = &x
ld a,(a) // a = x
ldi b,z // b = &z
ld b,(b) // b = z
add a,b // a = x+z
ldi b,z // b = &z
st (b),a // z = x+z
// }
while1\textit{while1}endif1:
// }
// }
// }
halt
x:
byte 0
y:
byte 0
z:
byte 0

Then we turn the conditional goto into native assembly language code:

// int x,y,z;
// int main(void)
// {
// x = 5;
ldi a,x
ldi b, 5
st (a),b
// z = 0;
ldi a,z
ldi b, 0 ;
st (a),b
// while (x > 0)
// {
// y = 0;
// while (y < 5)
// {
// if (((x > 2) && (x != y)) || (y == 1))
// if (!(((x > 2) && (x != y)) || (y == 1))) goto while1\textit{while1}endif1;
// if ((!((x > 2) && (x != y)) && !(y == 1))) goto while1\textit{while1}endif1;
// if ((x > 2) && (x != y)) goto while1\textit{while1}if1;
// if (!(x > 2)) goto while1\textit{while1}if1_l1;
// if ((x < 2) || (x == 2)) goto while1\textit{while1}if1_l1;
ldi a,x // a = &x
ld a,(a) // a = x
ldi b, 2 // b = 2
cmp a,b // a-b
jci while1\textit{while1}if1_l1
jzi while1\textit{while1}if1_l1
// if (x != y) goto while1\textit{while1}if1;
ldi a,x // a = &x
ld a,(a) // a = x
ldi b,y // b = &y
ld b,(b) // b = y
cmp a,b // a-b
jzi while1\textit{while1}if1_l1
jmp while1\textit{while1}if1
while1\textit{while1}if1_l1:
// if (!(y == 1)) goto while1\textit{while1}endif1;
ldi a,y // a = &y
ld a,(a) // a = y
Now is a good time to test the code with different combinations of values for the variables.

Once the conditional statement is checked, we can work on the inner loop.

Note the use of indentation makes it (relatively) easy to locate the beginning and the end of the

inner while loop. Next, we translate the inner loop to assembly code:

This is a good time to test the inner loop. Once the inner loop is tested (to a certain extent), it is

time to complete the outer loop. I am combining all the steps, as it gets boring after a while!

And that’s it!

This site is licensed under Creative Commons Attribution-NonCommercial-ShareAlike 4.0

International .

ldi b, 1 // b = 1
cmp a,b // a-b
jzi while1\textit{while1}if1
jmpi while1\textit{while1}endif1
// {
// z = z + x;
while1\textit{while1}if1:
ldi a,x // a = &x
ld a,(a) // a = x
ldi b,z // b = &z
ld b,(b) // b = z
add a,b // a = x+z
ldi b,z // b = &z
st (b),a // z = x+z
// }
while1\textit{while1}endif1:
// }
// }
// }
halt
x:
byte 0
y:
byte 0
z:
byte 0
\end{document}
