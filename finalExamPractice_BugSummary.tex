\documentclass[9pt]{article}
\usepackage[utf8]{inputenc}
\usepackage{geometry}
\geometry{a4paper, margin=0.4in, top=0.3in, bottom=0.3in}
\usepackage{listings}
\usepackage{xcolor}
\usepackage{hyperref}
\usepackage{booktabs}
\usepackage{longtable}
\usepackage{enumitem}
\usepackage{titlesec}
\usepackage{parskip}
\usepackage{fancyhdr}

% Remove header/footer spacing
\pagestyle{plain}
\setlength{\headheight}{0pt}
\setlength{\headsep}{0pt}
\setlength{\footskip}{10pt}

% Compact spacing for sections
\titlespacing*{\section}{0pt}{3pt}{1pt}
\titlespacing*{\subsection}{0pt}{2pt}{0.5pt}
\titlespacing*{\subsubsection}{0pt}{1.5pt}{0pt}
\titlespacing*{\paragraph}{0pt}{0.5pt}{0pt}

% Compact lists
\setlist{nosep, leftmargin=12pt, itemsep=0pt, parsep=0pt, topsep=1pt, partopsep=0pt}
\setenumerate{nosep, leftmargin=12pt, itemsep=0pt, parsep=0pt, topsep=1pt}
\setitemize{nosep, leftmargin=12pt, itemsep=0pt, parsep=0pt, topsep=1pt}

% Reduce paragraph spacing
\setlength{\parskip}{1pt}
\setlength{\parindent}{0pt}
\setlength{\baselineskip}{10pt}

% Compact code listings - preserves ALL comments
\lstset{
    basicstyle=\ttfamily\scriptsize,
    breaklines=true,
    frame=single,
    framesep=2pt,
    numbers=left,
    numbersep=3pt,
    numberstyle=\tiny\color{gray},
    xleftmargin=5pt,
    xrightmargin=5pt,
    aboveskip=2pt,
    belowskip=2pt,
    showstringspaces=false,
    keepspaces=true,
    tabsize=2,
    commentstyle=\color{gray!70}\itshape,
    keywordstyle=\color{blue}\bfseries,
    stringstyle=\color{red!70},
    escapeinside={@*}{*@},
}

\title{Finalexampractice Bugsummary}
\author{}
\date{}

\begin{document}

\maketitle
\vspace{-10pt}

\section{Bug Summary - Quick Reference}

\subsection{Critical Bugs Found and Fixed}

\subsubsection{1. **Line 60: Missing Stack Initialization**}
\begin{lstlisting}[language={[x86masm]Assembler}, commentstyle=\color{gray}\itshape]
// BUG:
ldi d,

// FIX:
ldi d, 0        // Initialize stack pointer to 0
\end{lstlisting}

\subsubsection{2. **Line 107: Wrong Type Size for fPtr**}
\begin{lstlisting}[language={[x86masm]Assembler}, commentstyle=\color{gray}\itshape]
// BUG:
isLoop_result: isLoop_fPtr Node_size +

// FIX:
isLoop_result: isLoop_fPtr 1 +  // fPtr is a pointer (1 byte), not a Node (2 bytes)
\end{lstlisting}

\subsubsection{3. **Line 112: Invalid Syntax for Allocation**}
\begin{lstlisting}[language={[x86masm]Assembler}, commentstyle=\color{gray}\itshape]
// BUG:
ldi c,0 isLoop_lvs -
add d,c

// FIX:
ldi c, isLoop_lvs  // Load local var size
sub d, c            // Allocate locals (subtract from stack pointer)
\end{lstlisting}

\subsubsection{4. **Lines 119-121: Incorrect Parameter Access**}
\begin{lstlisting}[language={[x86masm]Assembler}, commentstyle=\color{gray}\itshape]
// BUG:
ldi a,isLoop_pNode
ld a,(a)           // Tries to dereference offset, not parameter!

// FIX:
ldi a, isLoop_pNode // A = offset to parameter
add a, d            // A = address of parameter
ld a, (a)           // A = value of parameter
\end{lstlisting}

\subsubsection{5. **Lines 124-129: Incorrect prevFrame Assignment**}
\begin{lstlisting}[language={[x86masm]Assembler}, commentstyle=\color{gray}\itshape]
// BUG:
ldi a,isLoop_prevFrame
add a,d
ldi c,Frame_prevFrame Frame_pNode -  // Wrong calculation
add b,c

// FIX:
ldi a, isLoop_prevFrame // A = offset to prevFrame parameter
add a, d                // A = address of prevFrame parameter
ld a, (a)               // A = prevFrame value
ldi b, isLoop_f
add b, Frame_prevFrame  // B = offset to f.prevFrame
add b, d                // B = address of f.prevFrame
st (b), a               // f.prevFrame = prevFrame
\end{lstlisting}

\subsubsection{6. **Lines 136-137: Wrong Parameter Access (Struct Member vs Parameter)**}
\begin{lstlisting}[language={[x86masm]Assembler}, commentstyle=\color{gray}\itshape]
// BUG:
ldi a,Frame_prevFrame  // This is struct MEMBER offset, not parameter!
ld a,(a)               // Wrong!

// FIX:
ldi a, isLoop_prevFrame // A = offset to prevFrame PARAMETER
add a, d                // A = address of prevFrame parameter
ld a, (a)               // A = prevFrame parameter value
\end{lstlisting}

\subsubsection{7. **Lines 156-162: Wrong Offset for pNode Parameter**}
\begin{lstlisting}[language={[x86masm]Assembler}, commentstyle=\color{gray}\itshape]
// BUG:
ldi c,Frame_pNode 1 +  // This gives 1, not parameter offset!
add c,d                // C = D+1 (points to f.prevFrame, not pNode!)
ld c,(c)               // Loads wrong value

// FIX:
ldi c, isLoop_pNode // C = 5 (offset to pNode parameter)
inc c                // C = 6 (adjust for the push: dec d, st (d),c)
add c, d             // C = D+6 (address of pNode parameter)
ld c, (c)            // C = pNode parameter value
\end{lstlisting}

\subsubsection{8. **Line 162: Wrong Jump Condition**}
\begin{lstlisting}[language={[x86masm]Assembler}, commentstyle=\color{gray}\itshape]
// BUG:
jzi isLoop_then0_while0_begin  // Loops forever if fPtr->pNode == pNode!

// FIX:
jzi isLoop_then0_while0_end    // Exit loop when fPtr->pNode == pNode
\end{lstlisting}

\subsubsection{9. **Line 169: Wrong Register for Storing fPtr**}
\begin{lstlisting}[language={[x86masm]Assembler}, commentstyle=\color{gray}\itshape]
// BUG:
st (c),b  // Register c doesn't contain fPtr address here!

// FIX:
// After computing fPtr->prevFrame in register C:
// Register B contains address of fPtr variable (from earlier)
st (b), c  // Store new fPtr value back to fPtr variable
\end{lstlisting}

\subsubsection{10. **Line 172: Register c Doesn't Contain fPtr**}
\begin{lstlisting}[language={[x86masm]Assembler}, commentstyle=\color{gray}\itshape]
// BUG:
and c,c  // Testing wrong register - c was used for other things!

// FIX:
ldi c, isLoop_fPtr  // C = offset to fPtr variable
add c, d            // C = address of fPtr variable
ld c, (c)           // C = fPtr value
and c, c            // Test if fPtr == 0
\end{lstlisting}

\subsubsection{11. **Line 182: Wrong Offset for pNode in Recursive Call**}
\begin{lstlisting}[language={[x86masm]Assembler}, commentstyle=\color{gray}\itshape]
// BUG:
ldi b,isLoop_pNode 1 +  // 5+1=6, but after push it's wrong offset
add b,d
ld b,(b)

// FIX:
// After push: dec d, st (d),c (push &f)
// pNode was at D+5, now at D+6 (frame base moved from D to D+1)
ldi b, isLoop_pNode // B = 5
inc b               // B = 6 (adjust for push)
add b, d            // B = D+6 (address of pNode parameter)
ld b, (b)           // B = pNode parameter value
\end{lstlisting}

\subsubsection{12. **Line 218: Missing Value for prevFrame Argument**}
\begin{lstlisting}[language={[x86masm]Assembler}, commentstyle=\color{gray}\itshape]
// BUG:
ldi a,  // No value!

// FIX:
ldi a, 0  // Push 0 as prevFrame argument
\end{lstlisting}

\subsubsection{13. **Line 232: Missing Value for Argument Cleanup**}
\begin{lstlisting}[language={[x86masm]Assembler}, commentstyle=\color{gray}\itshape]
// BUG:
ldi b,  // No value!
add d,b

// FIX:
inc d  // Clean up first argument
inc d  // Clean up second argument
\end{lstlisting}

\subsubsection{14. **Line 199: Wrong Label Name**}
\begin{lstlisting}[language={[x86masm]Assembler}, commentstyle=\color{gray}\itshape]
// BUG:
isLoop_endif0:  // But jump at line 132 goes to isLoop_endif

// FIX:
isLoop_endif:  // Match the label name used in jzi
\end{lstlisting}

\hrule

\subsection{Common Patterns to Watch For}

\begin{enumerate}
\item \textbf{Parameter Access Pattern}:
\end{enumerate}
\begin{lstlisting}[language={[x86masm]Assembler}, commentstyle=\color{gray}\itshape]
   ldi reg, param_offset  // Load offset
   add reg, d             // Compute address (D is stack pointer)
   ld reg, (reg)          // Load value
\end{lstlisting}

\begin{enumerate}
\item \textbf{Local Variable Access Pattern}:
\end{enumerate}
\begin{lstlisting}[language={[x86masm]Assembler}, commentstyle=\color{gray}\itshape]
   ldi reg, local_offset  // Load offset (usually small, like 0, 1, 2...)
   add reg, d             // Compute address (D points to frame base)
   ld/st (reg), ...       // Access value
\end{lstlisting}

\begin{enumerate}
\item \textbf{Struct Member Access Pattern}:
\end{enumerate}
\begin{lstlisting}[language={[x86masm]Assembler}, commentstyle=\color{gray}\itshape]
   ldi reg, struct_base   // Load base address of struct
   add reg, member_offset // Add member offset (e.g., Frame_pNode = 0)
   ld/st (reg), ...       // Access member
\end{lstlisting}

\begin{enumerate}
\item \textbf{Stack Balance Check}:
\end{enumerate}
   - Before function call: count \texttt{dec d} (pushes)
   - After function call: count \texttt{inc d} (pops)
   - Should differ by exactly 1 (the return address, popped by callee)

\hrule

\subsection{Register Tracking Tips}

\begin{itemize}
\item \textbf{Always track what each register contains} at each point
\item \textbf{Write comments} showing the C code concept each register represents
\item \textbf{Check for overwrites} before using a register value
\item \textbf{Use temporary saves} on stack if needed during complex operations
\end{itemize}

\hrule

\subsection{Stack Frame Visualization}

For \texttt{isLoop} function:
\begin{verbatim}
After allocation:
D+6: prevFrame (parameter 2)
D+5: pNode (parameter 1)
D+4: return address
D+3: result (local variable)
D+2: fPtr (local variable)
D+1: f.prevFrame (member of local struct f)
D+0: f.pNode (member of local struct f) ← D points here
\end{verbatim}

Key insight: \textbf{Offsets are always relative to where D points after allocation!}


\end{document}
