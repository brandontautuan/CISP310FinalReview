\documentclass[9pt]{article}
\usepackage[utf8]{inputenc}
\usepackage{geometry}
\geometry{a4paper, margin=0.4in, top=0.3in, bottom=0.3in}
\usepackage{listings}
\usepackage{xcolor}
\usepackage{hyperref}
\usepackage{booktabs}
\usepackage{longtable}
\usepackage{enumitem}
\usepackage{titlesec}
\usepackage{parskip}
\usepackage{fancyhdr}

% Remove header/footer spacing
\pagestyle{plain}
\setlength{\headheight}{0pt}
\setlength{\headsep}{0pt}
\setlength{\footskip}{10pt}

% Compact spacing for sections
\titlespacing*{\section}{0pt}{3pt}{1pt}
\titlespacing*{\subsection}{0pt}{2pt}{0.5pt}
\titlespacing*{\subsubsection}{0pt}{1.5pt}{0pt}
\titlespacing*{\paragraph}{0pt}{0.5pt}{0pt}

% Compact lists
\setlist{nosep, leftmargin=12pt, itemsep=0pt, parsep=0pt, topsep=1pt, partopsep=0pt}
\setenumerate{nosep, leftmargin=12pt, itemsep=0pt, parsep=0pt, topsep=1pt}
\setitemize{nosep, leftmargin=12pt, itemsep=0pt, parsep=0pt, topsep=1pt}

% Reduce paragraph spacing
\setlength{\parskip}{1pt}
\setlength{\parindent}{0pt}
\setlength{\baselineskip}{10pt}

% Compact code listings - preserves ALL comments
\lstset{
    basicstyle=\ttfamily\scriptsize,
    breaklines=true,
    frame=single,
    framesep=2pt,
    numbers=left,
    numbersep=3pt,
    numberstyle=\tiny\color{gray},
    xleftmargin=5pt,
    xrightmargin=5pt,
    aboveskip=2pt,
    belowskip=2pt,
    showstringspaces=false,
    keepspaces=true,
    tabsize=2,
    commentstyle=\color{gray!70}\itshape,
    keywordstyle=\color{blue}\bfseries,
    stringstyle=\color{red!70},
    escapeinside={@*}{*@},
}

\title{Parameterandlocalvariables}
\author{}
\date{}

\begin{document}

\maketitle
\vspace{-10pt}

Module 0307: Parameters and local variables
About this module
Parameters on the stack
Local variables
A more mechanical way to access frame items
Exercise 1
Exercise 2
\begin{enumerate}
\item About this module
\end{enumerate}
Prerequisites: 0304 , 0306
Objectives: This module discusses how parameters are passed in assembly language code
in a way that is compatible with C code. Local variables are also discussed in this module
\begin{enumerate}
\item Parameters on the stack
\end{enumerate}
Because there can many parameters, and recursion is to be supported (along with multi-
threading), it is necessary to store parameters in a per-invocation basis, much like the return
address.
Let us examine the following caller code in C:
subtract( 3 , 5 );
The equivalent code in TTP assembly is as follows:
ldi a, 5
dec d
st (d),a // push 5
ldi a, 3
dec d
st (d),a // push 3
ldi a,retAddr
dec d
st (d),a // push return address
jmpi subtract
retAddr:
inc d
inc d // deallocate parameters
// now the stack should be balanced
Note how the parameters are pushed in reverse order. This means the second parameter is
pushed first, and then the first parameter. Another implication is that the first parameter has a
lower address on the stack than the second parameter.
Furthermore, also note how the return address is pushed last.
Here is the subroutine code in C:
int subtract (int x, int y)
{
return x-y;
}
The matching code in TTP assembly is as follows:
subtract: // entry point of subroutine
cpr c,d // c is a copy of the stack pointer
ldi a, 1 // offset to find parameter x
add c,a // now c is the address to x
ld a,(c) // now a is parameter x
inc c // now c is the address to y
ld b,(c) // now b is parameter y
sub a,b // perform the subtraction
// a already has the right result to return
ld b,(d) // d is never changed, still pointer
// to the return address
inc d // callee only deallocates the retAddr
// NOT the parameters!
jmp b // now return and continue in caller
It is important to leave register D alone and not to use it directly to compute the address of
parameters.
\begin{enumerate}
\item Local variables
\end{enumerate}
Local variables of a subroutine are allocated on a per invocation basis on the stack, much like
parameters. However, parameters are allocated and initialized by the caller, while local
variables are allocated by the called subroutine.
As a result, local variables have addresses that are below that of the return address. The
method to access local variables is the same as accessing parameters.
The main difficulty of accessing local variables is tracking the stack pointer as items are
pushed and popped. In many real architectures, there is a “frame point” that points to a fixed
place relative to items used by a subroutine. TTP is too primitive to dedicate yet another
register for this purpose.
\begin{enumerate}
\item A more mechanical way to access frame
\end{enumerate}
items
The parameters, return address, and local variables collectively make up the call frame of an
invocation of a function. From the perspective of the callee, the parameters, and return
address are already pushed on the stack at the entry point. However, a callee is responsible to
allocate stack space for its own local variables.
It is best to use symbolic names to reference the offset to items in a frame from where the
stack pointer points to after the entire frame is allocated. For example, consider the following C
code:
void someFunc (uint8\textit{t x, uint8}t y, uint8_t z)
{
uint8_t a,b,c;
}
The corresponding frame looks like the following:
offset from where SP points to what
6 z
5 y
4 x
3 return address
2 c
1 b
0 a
From the callee’s perspective, some label definitions can be used to first compute the offsets
to local variables:
someFunc_a: 0 // offset from where D points to to find local var a
someFunc\textit{b: someFunc}a 1 + // offset to local var b
someFunc\textit{c: someFunc}b 1 + // offset to local var c
someFunc\textit{localVarSize: someFunc}c 1 + // # bytes used for local vars
Parameters are also in a call frame, they can also have their offsets defined by labels. In this
example, the following label definitions can apply:
someFunc\textit{x: someFunc}localVarSize 1 + // add 1 to skip over the return address
someFunc\textit{y: someFunc}x 1 +
someFunc\textit{z: someFunc}y 1 +
The “shell” of this callee consists of the code to allocate and deallocate the local variables, as
well as the code to return to the caller:
someFunc: // entry point of the callee
ldi a,someFunc_localVarSize
sub d,a // allocate stack space for local variables
// ... code for the body of someFunc
ldi b,someFunc_localVarSize
add d,b // deallocate stack space for local variables
// the SP should now point to the return address
ld b,(d)
inc d // pop the return address to reg b
jmp b // continue execution at the caller
In the body of someFunc, accessing a local variable or parameter can be very mechanical.
Generally, the following pattern works:
ldi a,someFunc_b // a=&b-SP load offset to a register
add a,d // a=&b the address of the item on the frame
Whether the address of a frame item should be dereferenced depends on the context.
It is important to keep track of the stack pointer using this approach! If the stack pointer is not
pointing to the local variable of the lower address, then adjustments need to be made. This
usually happens in a function call. Let’s reexamine someFunc with a call to someOtherFunc:
void someFunc (uint8\textit{t x, uint8}t y, uint8_t z)
{
uint8_t a,b,c;
// ...
someOtherFunc(&a,z);
// ...
}
The code to call someOtherFunc is then as follows
\begin{enumerate}
\item Exercise 1
\end{enumerate}
Implement the following program in TTP assembly language:
\section{include \textless{}stdint,h\textgreater{}}
void swap (uint8\textit{t \textit{pX, uint8}t }pY);
int main ()
{
uint8_t x,y;
x= 2 ;
y= 5 ;
swap(&x,&y);
return 0 ;
}
void swap (uint8\textit{t \textit{pX, uint8}t }pY)
ldi a,someFunc_z // a=&z-SP
add a,d // a=&z
ld a,(a) // a=z
dec d
// at this point, the SP points to one byte below local var a!
st (d),a // push z
ldi a,someFunc_a 1 + // reg a=&(local var a)-SP, add 1 to compensate for the second argument already pushed
add a,d // reg a=&(local var a)
dec d
st (d),a // push &(local var a)
ldi a,. 6 +
dec d
st (d),a // push return address
inc d // deallocate the first argument
inc d // deallocate the second argument
{
uint8_t t;
t=*pX;
\textit{pX=}pY;
*pY=t;
}
In addition to the use of local variables, this program also involves passing the address of a
variable (instead of the value of a variable) on the stack.
\begin{enumerate}
\item Exercise 2
\end{enumerate}
Implement the following program in TTP assembly language:
\section{include \textless{}stdint.h\textgreater{}}
uint8\textit{t power (uint8}t n);
int main ()
{
uint8_t x;
x = power( 3 );
return 0 ;
}
uint8\textit{t power (uint8}t n)
{
if (!n)
{
return 1 ;
}
else
{
return 2 *power(n-1);
}
}
The function in this program is recursive!
\end{document}
